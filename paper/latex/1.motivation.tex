\section{Motivation}
\label{sec:intro}

By 2025, it is predicted that 6.2 billion private motorized trips will be made all around the world every day \footnote{\url{https://goo.gl/7YcmFW}}. This means more roads, more users and many more security concerns. Road traffic monitoring is not just a huge challenge, but a necessity, and computer vision is playing an important role in addressing this problem. In this work we present a real-time road traffic monitoring system using a regular video camera mounted on a relatively high place, i.e. a bridge, with a significant image analysis field. The system is able to track and count vehicles, raising an alarm whenever any of them exceeds the speed limit.\\

\noindent The remaining of this paper is organized as follows. In section \ref{sec:related} different road traffic monitoring methods are presented. In section \ref{sec:method} we present the strategy followed to address the challenge of road traffic surveillance studied along this research. In section \ref{sec:evaluation} the results of our approach are presented and discussed. Finally, in section \ref{sec:conclusions} the conclusions are drawn along with the directions for future research.